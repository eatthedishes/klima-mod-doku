\documentclass[]{report}
\usepackage[utf8]{inputenc}


% Title Page
\title{Nulldimensionale Modelle zur Beschreibung globaler Klimaänderungen}
\author{Claudio Harringer 0927850 \\ Andreas Cremer 0926918}


\begin{document}
\maketitle
\tableofcontents
%\begin{abstract}
%\end{abstract}
\chapter{Einleitung} 
	Klima ist das Verhalten des Wetters unserer Atmosphäre über längerer Zeiträume. Hiermit sind Zeiträume in der Länge von Jahrzehnten gemeint, im Gegensatz zur kurzfristigen Wettervorhersage, die sich in der Größenordnung von Tagen oder Stunden bewegt. Klima wird im Allgemeinen über statistische Variablen (zum Beispiel Mittelwerte oder Varianzen) ausgedrückt. So beschriebene Variablen werden Klimaelemente genannt; dies sind unter anderem Temperatur, Niederschlag, Luftfeuchtigkeit, atmosphärischer Druck, Wind, Albedo, Bewölkung sowie Ein- und Ausstrahlung.\\
	Die Größe des System und mit ihr die Vielzahl der Einflussfaktoren und Variablen, die das Klimasystem der Erde beeinflussen und in ihm miteinander interagieren, machen eine erschöpfende Beschreiben desselben und damit eine exakte Vorhersage zukünftiger Entwicklungen unmöglich. Um dennoch verwendbare Beschreibungen des Klimas und realistische Vorhersagen seiner Veränderungen treffen zu können, müssen starke Vereinfachungen vorgenommen werden.\\
	Die Genauigkeit kurzfristiger Wettervorhersagen ist leicht überprüfbar. Erstens sieht man nach wenigen Tagen, ob die Vorhersagen (bzw. welche davon) eingetroffen sind und kann mithilfe dieses Wissens das Modell verfeinern. Zweitens kann man diesen Vorgang sehr oft wiederholen. Dies ist bei langfristigen Klimavorhersagen nicht möglich. Solche Betrachtungen sind somit um einiges schwieriger anzustellen.
	


\end{document}          
