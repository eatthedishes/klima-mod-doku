\documentclass[]{report}
\usepackage[utf8]{inputenc}
\usepackage{amsmath}
\usepackage{graphicx}


% Title Page
\title{Nulldimensionale Modelle zur Beschreibung globaler Klimaänderungen}
\author{Claudio Harringer 0927850 \\ Andreas Cremer 0926918}


\begin{document}
\maketitle
\tableofcontents
%\begin{abstract}
%\end{abstract}
\chapter{Einleitung} 
	Klima ist das Verhalten des Wetters unserer Atmosphäre über längerer Zeiträume. Hiermit sind Zeiträume in der Länge von Jahrzehnten gemeint, im Gegensatz zur kurzfristigen Wettervorhersage, die sich in der Größenordnung von Tagen oder Stunden bewegt. Klima wird im Allgemeinen über statistische Variablen (zum Beispiel Mittelwerte oder Varianzen) ausgedrückt. So beschriebene Variablen werden Klimaelemente genannt; dies sind unter anderem Temperatur, Niederschlag, Luftfeuchtigkeit, atmosphärischer Druck, Wind, Albedo, Bewölkung sowie Ein- und Ausstrahlung.\par\bigskip
	Die Größe des System und mit ihr die Vielzahl der Einflussfaktoren und Variablen, die das Klimasystem der Erde beeinflussen und in ihm miteinander interagieren, machen eine erschöpfende Beschreiben desselben und damit eine exakte Vorhersage zukünftiger Entwicklungen unmöglich. Um dennoch verwendbare Beschreibungen des Klimas und realistische Vorhersagen seiner Veränderungen treffen zu können, müssen starke Vereinfachungen vorgenommen werden.\par\bigskip
	Die Genauigkeit kurzfristiger Wettervorhersagen ist leicht überprüfbar. Erstens sieht man nach wenigen Tagen, ob die Vorhersagen (bzw. welche davon) eingetroffen sind und kann mithilfe dieses Wissens das Modell verfeinern. Zweitens kann man diesen Vorgang sehr oft wiederholen. Dies ist bei langfristigen Klimavorhersagen nicht möglich. Solche Betrachtungen sind somit um einiges schwieriger anzustellen. \par\bigskip
	Modelle zur Klimabeschreibung lassen sich nach der Anzahl der betrachteten räumlichen Dimensionen klassifizieren:
	\begin{itemize}
		\item Nulldimensionale Klimamodelle haben keine Räumliche Dimension, es werden uniforme Bedingungen für den gesamten Planeten angenommen. Die Variablen sind Funktionen der Zeit.
		\item In eindimensionalen Klimamodellen wird üblicherweise die geographische Breite berücksichtigt. Dadurch können Wärmeflüsse zwischen Äquatorial- und Polarregionen simuliert werden.
		\item In höherdimensionalen Modellen können beispielsweise Atmosphärenschichten simuliert werden. Allerdings können solche Modelle normalerweise nicht mehr analytisch behandelt werden.
	\end{itemize}
	Im Folgenden beschränken wir uns auf zwei nulldimensionale Klimamodelle, das Modell nach Griffel und Drazin und das nach Fraedrich.
	
	
	
	
	
	
	
	
\chapter{Modelle}

\section{Gemeinsame Voraussetzungen}

Ein wichtiger Faktor im Klimasystem der Erde und in unseren Modellen ist die planetare Albedo. Albedo ist ein Maß für das Rückstrahlungsvermögen von Oberflächen. Sie ist eine dimensionslose Zahl und stellt das Verhältnis von reflektierter und einfallender Energie dar. Die Albedo der Erde beträgt 0,367. Veränderungen in der Eisbedeckung der Ozeane haben einen großen Einfluss auf die planetare Albedo. Somit ist diese stark temperaturabhängig. Kältere Temperaturen sorgen für mehr Meereis und somit für eine höhere Albedo. Höhere Temperaturen sorgen für eisfreie Ozeane und somit weniger Reflexion. Dies sorgt für einen selbst verstärkenden Effekt auf Temperaturveränderungen, da höhere Albedo dadurch, dass mehr Energie wieder ins All zurückgestrahlt wird, die Temperatur weiter verringert. Dieser Effekt ist in beiden Modellen zu beobachten: Beide haben einen instabilen Gleichgewichtspunkt, ab dem Temperaturen zu einem niedrigen stabilen Gleichgewicht fallen.\par\bigskip

Die beiden hier behandelten Modelle sind Differentialgleichungen der Temperatur in der Zeit. Sie verwenden viele gemeinsame Parameter.\\

\begin{itemize}
	\item $T$ ist die (zeitlich und räumlich gemittelte)Temperatur in Kelvin
	\item $\epsilon$ ist die Bewölkungskonstante. Sie gibt die Abweichung vom einem schwarzen Körper mit $\epsilon=1$ an. Bei Fraedrich ist $\epsilon = 0.69$
	\item $Q = 430 \frac{W}{m^2}$ ist die auf der Erde eintreffende Strahlung.
	\item $\mu$ ist ein Faktor, mit dem $Q$ multipliziert wird, um Veränderungen im solaren Strahlungsangebot zu simulieren. Somit ist im Normalfall $\mu = 1$.
	\item $\sigma = 5,67*10^{-8} \frac{W}{m^2K^4}$ ist die Stefan-Boltzmann-Konstante.
	\item $h = 75m$ ist die Tiefe des Klimasystems, bis zu der Strahlunsenergieumsätze stattfinden.
	\item $\rho = 10^3 \frac{kg}{m3}$ ist die Dichte der Atmosphäre.
%	\item c_w
 
\end{itemize}





\section{Griffel und Drazin}
	\begin{align}
		\dfrac{dT}{dt} = (0,5 \tanh\left(\frac{T^6}{T_0^6}\right)-1) \sigma T^4 + \mu Q (0,58 + 0,2 \tanh (0,052 (T - 276,15)))
		%((0.5 * np.tanh(var_T ** 6 / var_T0 ** 6) - 1) * var_sigma * var_T ** 4 + var_mu * var_Q * (0.58 + 0.2 * np.tanh(0.052 * (var_T - 276.15)))) / var_c
	\end{align}
\section{Fraedrich}

Die von Fraedrich verwendeten Gleichungen brauchen noch zwei weitere Konstanten: $a=2,8$ und $b=0,009 K^{-1}$.
	\begin{align}
		p=\frac{\mu Q b}{\epsilon \sigma}
	\end{align}
	\begin{align}
		q = - \frac{\mu Q (1-a)}{\epsilon \sigma}
	\end{align}
	Fraedrich verwendet bei und über 227,78 K eine lineare Albedo-Temperatur-Relation, dadurch ergibt sich folgende Differentialgleichung:
	\begin{align}
		f_{oben}(T) = \frac{\epsilon \sigma}{c} (-T^4 + p T - q)
	\end{align}
	Unter 227,78 K nimmt er die Albedo als konstant 0,75 an, die Gleichung vereinfacht sich hier also:
	\begin{align}
		f_{unten}(T) = \frac{\epsilon \sigma}{c} (-T^4 + \frac{p}{4b})
	\end{align}
	

\chapter{Fallbeispiele}

% fr320 http://biomath.pythonanywhere.com/run/fraedrich/320?dl
\begin{figure} \centering \def\svgwidth{\columnwidth} \input{fr320.pdf_tex} \label{fr320} \end{figure}

In Abbildung \ref{fr320} starten wir mit einer Temperatur deutlich über dem oberen stabilen Gleichgewicht. Die Temperatur fällt schnell in Richtung Gleichgewicht und verbleibt dort. Im Modell nach Griffel und Drazin wären die Ergebnisse ähnlich.

% gd288 http://biomath.pythonanywhere.com/run/griffeldrazin/288.7?dl
\begin{figure} \centering \def\svgwidth{\columnwidth} \input{gd288.pdf_tex} \end{figure}

Dies ist ein Startwert knapp unter dem instabilen Gleichgewicht. Die Temperatur verändert sich erst nur sehr langsam und pendelt sich erst spät beim niedrigeren Gleichgewicht ein. Hier unterscheidet sich das Modell von dem Modell nach Fraedrich, dort wären wir beim hohen Gleichgewicht gelandet.





\section{Stabilitätsanalyse}

% stabfr http://biomath.pythonanywhere.com/stabanalysis/fraedrich?dl
\begin{figure} \centering \def\svgwidth{\columnwidth} \input{stabfr.pdf_tex} \label{stabfr} \end{figure}
% stabgd http://biomath.pythonanywhere.com/stabanalysis/griffeldrazin?dl
\begin{figure} \centering \def\svgwidth{\columnwidth} \input{stabgd.pdf_tex} \label{stabgd} \end{figure}

In diesen beiden Plots \ref{stabfr} und \ref{stabgd} ist das Stabilitätsverhalten der Modelle dargestellt. Die Kurve stellt die Gleichgewichte der Temperatur in Abhängigkeit von Vielfachen des derzeitigen Strahlungsangebotes (also die auf der Erde eintreffende Sonnenenergie) dar. Die Veränderliche ist hierbei $\mu$, wobei $\mu = 1$ dem derzeit auf der Erde vorhandenen Strahlungsangebot entspricht. $\mu = 1$ ist in den Plots durch die vertikale schwarze Linie gekennzeichnet. Die Pfeile zeigen die Trajektorien der Temperatur ausgehend von verschiedenen Anfangszuständen an. Hieraus kann man leicht stabile und instabile Gleichgewichte ablesen: Wo drei Gleichgewichte vorhanden sind, sind die beiden äußeren stabil und das innere abstoßend.\\
Beide Modelle enthalten zwei Bifurkationspunkte. für $\mu$ etwas kleiner als $1$ verschwinden die oberen Gleichgewichte. Falls $\mu$ soweit sinkt, bewegt sich die Temperatur also in Richtung des verbleibenden niedrigen Gleichgewichtspunkts. Dieser würde nur wieder verlassen, falls $\mu$ im Anschluss stark bis zum zweiten Bifurkationspunkt stiege. Da eine so hohe Schwankung eher unwahrscheinlich ist, verbliebe nach beiden Modellen eine einmal erreichte "Schneeballerde" höchstwahrscheinlich dauerhaft in einem kalten Zustand.

% comp http://biomath.pythonanywhere.com/comparison?dl
\begin{figure} \centering \def\svgwidth{\columnwidth} \input{comp.pdf_tex} \label{comp} \end{figure}

Hier in Abbildung \ref{comp} ist zu sehen, dass die beiden Modelle ein vom Konzept her sehr ähnliches Verhalten aufweisen. Das Modell nach Fraedrich ist weniger anfällig gegen eine Verringerung des Strahlungsangebotes und das Entkommen aus einer Schneeballerde ist etwas leichter.

\end{document}          
